% !TeX root = ../main.tex

\begin{acknowledgements}

从研究课题的确定到最终完成毕业论文,前前后后大约过去了近半年的时间。
在这半年当中,有太多值得铭记的时刻:2022年的12月8日,科大终于解封了,回归了久违的正常生活;
2022年的12月11日,提交完了申请季最后一所学校的网申表;2023年1月24日,收到了EPFL的无面试录取——申请季的第一封录取信;2023年2月1日,
去上海领事馆面签,顺利通过;2023年2月10日,收到了普林斯顿大学的录取,申请季正式结束;
2023年3月25日,拖着尚未痊愈的身体,独自一人启程飞跃重洋;2023年4月11日,第一次在组会中作报告,介绍自己的工作;
2023年4月21日,第一次参加了学术研讨会;2023年4月30日,也就是今天,四月的最后一天,我终于完成了自己的毕业论文。

我一直觉得,自己是极度幸运的:在今年的申请季,作为一个转专业的选手,基础不甚扎实的情况下,仍然被普林斯顿录取,实在是无法用“实力足
够”来进行解释;在UCI的项目报名截止以后,仍然通过国合部的老师帮忙完成了报名;在Irvine的这些日子里,老师同学们的热情善良让在异
国他乡的我无比温暖;更不用说前三年在科大的时光,我收到了多少的鼓励与帮助……因此,我十分珍惜我现
在拥有的一切,希望自己能够对得起自己的运气,不辜负如此难得的机会。

在自己的科研学习过程中,我收到了很多老师和同学的帮助。我首先想感谢的是金西老师,正是他让我明确了自己的人生目标,知道了“未来到底需
要做什么”,帮助我从零开始迈入科研的大门;我还想感谢我的暑研老师——佐治亚理工学院的郝聪老师,她善良乐观的性格深深地影响着我,她还身
体力行的告诉我要成为一个“学术社牛”,多去结交朋友,多去建立人脉,共同在学术上产生新的突破;另外我想感谢的就是加州大学尔湾分校的黄思
陶老师,在我做毕设期间给予了全力的支持和帮助:他将最好的加
速卡给我使用,每周还会跟我聊天,探讨毕设中所遇到的问题,真的让我非常感动。另外,SoC设计室的学长袁伟、贵雨宸、佐治亚理工学院博士
生陈涵秋、Rishov Sakar、伊利诺伊大学厄巴纳-香槟分校博士生李煜鸿、杜克大学博士生李苇航、加州大学尔湾分校博士生徐浩成、徐嘉涔等都
在我申请或是科研过程当中给予了相当重要的帮助,在此一并感谢。

最后还想感谢我的家人们,感谢你们对我无限的爱与包容,我永远爱你们。

\end{acknowledgements}
