% !TeX root = ../main.tex

\begin{notation}

  % \begin{notationlist}{2em}
  %   \item[II]  启动时间间隔(Initiation Interval)。指流水线完全填满之前需要的初始化周期。
  %   \item[$O$] 渐进符号,用于描述函数的渐进行为。
  % \end{notationlist}
\linespread{1.5}
\begin{table}[htbp]
\centering

\resizebox{\linewidth}{!}{
\begin{tabular}{ccc}
\toprule
\textbf{英文缩写} & \textbf{英文全称}                           & \textbf{中文释义}       \\
\midrule
\textbf{FPGA} & Field Programmable Gate Arrays & 现场可编程逻辑门阵列 \\
\textbf{HLS}  & High Level Synthesis           & 高层次综合      \\
\textbf{DRAM} & Dynamic Random Access Memory   & 动态随机存储器    \\
\textbf{HBM}  & High Bandwidth Memory          & 高带宽存储器     \\
\textbf{LUT}  & Look Up Table                  & 查找表        \\
\textbf{DSP}  & Digital Signal Processor       & 数字信号处理器    \\
\textbf{II}   & Initiation Interval            & 启动时间间隔    \\
\bottomrule
\end{tabular}}
\end{table}
\end{notation}



% 也可以使用 nomencl 宏包

% \printnomenclature

% \nomenclature{$\displaystyle a$}{The number of angels per unit are}
% \nomenclature{$\displaystyle N$}{The number of angels per needle point}
% \nomenclature{$\displaystyle A$}{The area of the needle point}
% \nomenclature{$\displaystyle \sigma$}{The total mass of angels per unit area}
% \nomenclature{$\displaystyle m$}{The mass of one angel}
% \nomenclature{$\displaystyle \sum_{i=1}^n a_i$}{The sum of $a_i$}
